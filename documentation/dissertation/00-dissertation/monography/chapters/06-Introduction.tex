\chapter{Introdução}
	\begin{myenv}{1.5}
		\section{Considerações Iniciais}
			\par Os \textit{voice spoofing attacks} do tipo \textit{playback speech} constituem o tema de estudo deste trabalho.
			\par Para a realização desta pesquisa coletou-se uma série de vozes nos arredores do Instituto de Biociências, Letras e Ciências Exatas em São José do Rio Preto no estado de São Paulo. Tais gravações se constituem de dígitos em um intervalo de 0 a 9 falados tanto em língua inglesa como em portuguesa totalizando assim uma base com 820 trechos de áudio.
						 
		\section{Objetivos}
			\par No presente momento muito poucos artigos tratam da detecção de \textit{voice spoofing} usando a tecnologia de \textit{wavelets/wavelets packet} apesar de sua promissora capacidade em analisar de forma detalhada tanto as baixas quanto as altas frequências de um sinal.
			\par A proposta desse trabalho é tentar cobrir um pouco dessa deficiência usando wavelets em conjunção com tecnologias já estabelecidas como análise de espectro baseado na escala MEL ou BARK e análise paraconsistente de características.
			
		\section{Estrutura do trabalho}
			\par No capítulo \ref{chap:revBibli} se realizará uma revisão dos seguintes conceitos:
			\begin{itemize}
				\item Engenharia paraconsistente.
				\item Filtros digitais usando wavelets.
				\item Caracterização dos processos de produção da voz humana.
				\item Amostragem, quantização, entre outros.
			\end{itemize}
			\par Em seguida no capítulo \ref{chap:propApproach} apresentar-se-á a proposta de aproximação do problema usada nesta investigação. 
			\par No capítulo \ref{chap:testsResults} são mostrados os testes e resultados obtidos para, no capítulo \ref{chap:conclusions} se apresentar as conclusões.
			\par Ao final do documento se encontram as referências usadas.
		
		
	\end{myenv}