\setlength{\parindent}{0pt}
\newpage \thispagestyle{empty}
\vspace{1.5cm}
\fontsize{12}{\baselineskip} \selectfont

\begin{center}
	{\huge{\textbf{RESUMO}}}
\end{center}

\begin{myenv}{1.5}
	\fontsize{12}{\baselineskip} \selectfont \onehalfspacing
	\par \null
	\par \null
	\par \textit{Voice spoofing} é uma técnica que tenta ludibriar sistemas de segurança baseados em identificação por voz. Este trabalho visa inicialmente, através da decomposição do sinal com wavelets e posterior cálculo da energia em intervalos baseados na escala BARK ou MEL, determinar qual a melhor combinação BARK/MEL-wavelet para que se obtenha uma separação máxima entre duas classes (voice spoofing e audio original) usando análise paraconsistente de características. Num segundo momento, após apuração da melhor combinação, os vetores de características gerados pelas técnicas selecionadas são submetidos a centenas de ensaios para classificação mudando-se o tamanho do conjunto de treinamento e testes de forma que, a cada novo teste, esses conjuntos são misturados de forma aleatória. Os classificadores usados foram os de distância Euclidiana e Manhattan além de Máquina de suporte de vetores. A acurácia máxima conseguida na distância Euclidiana foi de 90\%, na distância Manhattan de 91\% e usando Máquinas de suporte de vetores 95\%.
\end{myenv}

\setlength{\parindent}{0pt}
\newpage \thispagestyle{empty}
\vspace{1.5cm}
\fontsize{12}{\baselineskip} \selectfont
\begin{center}
	{\huge{\textbf{ABSTRACT}}}
\end{center}

\begin{myenv}{1.5}
	\fontsize{12}{\baselineskip} \selectfont \onehalfspacing
	\par \null
	\par \null
	\par \textbf{CORRIGIR!!!} Voice spoofing is a technique that attempts to circumvent security systems based on voice identification. This work initially aims, through the decomposition of the signal with wavelets and later calculation of the energy in intervals based on the BARK or MEL scale, to determine which is the best combination BARK/MEL with wavelet so that a maximum separation between two signal classes is obtained (i.e. voice spoofing and original audio). For that, the techniques of paraconsistent analysis of characteristics were used. In a second step, after determining the best combination, the feature vectors generated by the selected techniques were subjected to hundreds of classification cycles, changing the sizes of the training sets and tests so that, with each new test, these sets were mixed at random. The classifiers used were those of Euclidean and Manhattan distance and Support Vector Machine. The maximum accuracy achieved at the Euclidean distance was 90\%, at the Manhattan distance at 91\% and using Support Vector Machines 95\%.
\end{myenv}