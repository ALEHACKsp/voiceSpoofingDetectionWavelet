\chapter{Introdução}
	\begin{myenv}{1.5}
		\setcounter{page}{12}
		\par Os sistemas de autenticação de usuário por voz e o processo de falseamento usando \textit{voice spoofing} é o tema de estudo que se pretende explorar. Um sistema de reconhecimento \textbf{idealmente} não deve se deixar enganar por, como exemplo, uma voz gravada, neste documento será apresentada uma revisão de conceitos que visa preparar as bases para a construção de métodos que reconheçam esse ataque.
		
		\par No capítulo 2 se realizará uma revisão dos seguintes conceitos:
		\begin{itemize}
			\item Engenharia paraconsistente.
			\item Filtros digitais usando wavelets.
			\item Caracterização dos processos de produção da voz humana.
			\item Amostragem, quantização, entre outros.
		\end{itemize}
		\par Em seguida apresentar-se-á uma revisão bibliográfica finalizando com o cronograma previsto.
	\end{myenv}