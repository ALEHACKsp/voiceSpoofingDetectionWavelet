\setlength{\parindent}{0pt}
\newpage \thispagestyle{empty}
\vspace{1.5cm}
\fontsize{12}{\baselineskip} \selectfont

\begin{center}
	{\huge{\textbf{RESUMO}}}
\end{center}

\begin{myenv}{1.5}
	\fontsize{12}{\baselineskip} \selectfont \onehalfspacing
	\par \null
	\par \null
	\par \textit{Voice spoofing} é uma técnica que tenta ludibriar sistemas de segurança baseados em identificação por voz. Este trabalho visa inicialmente, através da decomposição do sinal com wavelets e posterior cálculo da energia em intervalos baseados na escala BARK ou MEL, determinar qual a melhor combinação BARK/MEL-wavelet para que se obtenha uma separação máxima entre duas classes (voice spoofing e audio original) usando análise paraconsistente de características. Após apuração da melhor combinação, os vetores de características gerados são submetidos a ensaios para classificação mudando-se o tamanho do conjunto de treinamento e testes de forma que, para cada novo teste, esses conjuntos são misturados de forma aleatória. Os classificadores usados foram os de distância Euclidiana e Manhattan além de Máquina de suporte de vetores(SVM). A acurácia máxima conseguida na distância Euclidiana e Manhattan foi de 90,97\% e usando SVM 95,12\%.\\\\
	Palavras-chave: voice spoofing, wavelet, análise paraconsistente.
\end{myenv}

\setlength{\parindent}{0pt}
\newpage \thispagestyle{empty}
\vspace{1.5cm}
\fontsize{12}{\baselineskip} \selectfont
\begin{center}
	{\huge{\textbf{ABSTRACT}}}
\end{center}

\begin{myenv}{1.5}
	\fontsize{12}{\baselineskip} \selectfont \onehalfspacing
	\par \null
	\par \null
	\par \textit{Voice spoofing} is a technique that attempts to circumvent security systems based on voice identification. This work initially aims, through the decomposition of the signal with wavelets and later calculation of the energy in intervals based on the BARK or MEL scale, to determine which is the best BARK/MEL-wavelet combination to obtain a maximum separation between two classes (voice spoofing and audio) using paraconsistent feature analysis. After determining the best combination, the feature vectors generated are subjected to tests for classification, changing the size of the training set and tests so that, for each new test, these sets are mixed randomly. The classifiers used were those of Euclidean and Manhattan distance as well as Support vector machine(SVM). The maximum accuracy achieved at Euclidean and Manhattan distance was 90.97\% and using SVM 95.12\%.\\\\
	Key-words: voice spoofing, wavelet, paraconsistent analysis.
\end{myenv}






