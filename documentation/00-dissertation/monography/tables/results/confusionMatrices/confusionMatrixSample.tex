\begin{table}[H]
	\newcommand{\mc}[3]{\multicolumn{#1}{#2}{#3}}
	\definecolor{tcB}{rgb}{0.447059,0.74902,0.266667}
	\definecolor{tcC}{rgb}{0,0,0}
	\definecolor{tcD}{rgb}{0,0.5,1}
	\definecolor{tcA}{rgb}{0.65098,0.65098,0.65098}
	\begin{center}
		\caption{Exemplo de matriz de confusão.}
		\begin{tabular}{ccc}
			% use packages: color,colortbl
			\mc{1}{l}{} & \mc{1}{>{\columncolor{tcA}}c}{\textbf{Verdadeiro}} & \mc{1}{>{\columncolor{tcA}}c}{\textbf{Falso}}\\
	
			\mc{1}{>{\columncolor{tcA}}r}{\textbf{Verdadeiro}} & \mc{1}{>{\columncolor{tcB}}c}{\textcolor{tcC}{TP}} & \mc{1}{>{\columncolor{tcD}}c}{\textcolor{tcC}{FP}}\\
	
			\mc{1}{>{\columncolor{tcA}}r}{\textbf{Falso}} & \mc{1}{>{\columncolor{tcD}}c}{\textcolor{tcC}{FN}} & \mc{1}{>{\columncolor{tcB}}c}{\textcolor{tcC}{TN}}
		\end{tabular}
		\label{tab:confusionMatrixSample}
		\\Fonte: Elaborado pelo autor, 2021.
	\end{center}
\end{table}