\begin{lstlisting}[language=C++, caption={Algoritmo para o cálculo do EER}, label={lst:EERAlgo}]
double menorDistanciaDaReta = valorMaximo();
double distanciaAtual = -valorMaximo();

double coordenadaAcimaDaRetaDiagonal = { 0, 0 };
double coordenadaAbaixoDaRetaDiagonal = { 0, 0 };

ordenarCrescente(vetorFAR);
ordenarDecrescente(vetorFRR);

for (i = 0; i < vetorFAR.size(); i++) {
	// Calcula a distancia entre a coordenada definida pelos valores de FAR e FRR
	// e a reta definida pela equacao x=y
	distanciaAtual = valorAbsoluto((vetorFAR[i] - vetorFRR[i]) / raizQuadrada(2));
	
	// Armazena apenas a menor distancia dos pontos acima da reta x=y
	if (distanciaAtual <= menorDistanciaDaReta && vetorFRR[i] >= vetorFAR[i]) {
		menorDistanciaDaReta = distanciaAtual;
		coordenadaAcimaDaRetaDiagonal[0] = vetorFAR[i];
		coordenadaAcimaDaRetaDiagonal[1] = vetorFRR[i];

		pularIteracao();
	}

	// Armazena apenas a menor distancia dos pontos abaixo da reta x=y
	coordenadaAbaixoDaRetaDiagonal[0] = vetorFAR[i];
	coordenadaAbaixoDaRetaDiagonal[1] = vetorFRR[i];
	
	paraIteracao();
}

// Calcula o ponto de interseccao da curva com a reta x=y
y = coordenadaAcimaDaRetaDiagonal[0] - coordenadaAbaixoDaRetaDiagonal[0] - 1;
x = coordenadaAbaixoDaRetaDiagonal[1] - coordenadaAcimaDaRetaDiagonal[1] + 1;
c = coordenadaAcimaDaRetaDiagonal[1] * coordenadaAbaixoDaRetaDiagonal[0] - coordenadaAcimaDaRetaDiagonal[0] * coordenadaAbaixoDaRetaDiagonal[1];

EER = -(c / (x + y));
\end{lstlisting}