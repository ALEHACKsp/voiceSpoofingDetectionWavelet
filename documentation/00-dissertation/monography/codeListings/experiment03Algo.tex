\begin{lstlisting}[language=C++, caption={Algoritmo que caracteriza o procedimento 03}, captionpos=t, label={lst:experiment03Algo}, belowskip=-0.8 \baselineskip]
tamanhosDoModelo={0.1, 0,2, 0,4, 0,5};
modeloDeReferenciaGenuino={};
modeloDeReferenciaFalseado={};
testesGenuino={};
testesFalseado={};
listaDeAcuracias={};

for(porcentagem : tamanhosDoModelo){
	for(teste = 0; teste < 300; teste++){
		// Escolhe aleatoriamente os sinais para o modelo com falseamento 
		// e os grava em 'modeloDeReferenciaFalseado' o restante vai 
		// para 'testesFalseado'
		escolherAleatoriamente(listaComFalseamento, porcentagem, modeloDeReferenciaFalseado, testesFalseado);
		
		// Escolhe aleatoriamente os sinais para o modelo sem spoofing
		// e os grava em 'modeloDeReferenciaGenuino' o restante vai 
		// para 'testesGenuino'
		escolherAleatoriamente(listaGenuino, porcentagem, modeloDeReferenciaGenuino, testesGenuino);
		treinarClassificador("falseado", modeloDeReferenciaFalseado);
		treinarClassificador("genuino", modeloDeReferenciaGenuino);
		
		// Classifica os testes e preenche a tabela de confusao
		for(sinal : testesFalseado){
			preencherTabelaDeConfusao(sinal, "falseado");
		} 
		
		// Classifica os testes e preenche a tabela de confusao
		for(sinal : testesGenuino){
			preencherTabelaDeConfusao(sinal, "genuino");
		}
		
		acuracia=calculaAcuracia();
		
		// Guarda as acuracias para cada porcentegem
		listaDeAcuracias[porcentagem].adicionar(acuracia);
		
		// Salva a melhor acuracia e matriz de confusao
		if(ehAMelhorAcuracia(acuracia)){
			salvaAcuraciaEMatrizDeConfusao();
		}
		
		// Salva a pior acuracia e matriz de confusao
		if(ehAPiorAcuracia(acuracia)){
			salvaAcuraciaEMatrizDeConfusao();
		}
	}
	
	// Calcula e salva o desvio padrao para cada porcentagem
	calculaESalvaDesvioPadrao(listaDeAcuracias[porcentagem]);
}
\end{lstlisting}
\begin{center}
	\par Fonte: Elaborado pelo autor, 2021.
\end{center}