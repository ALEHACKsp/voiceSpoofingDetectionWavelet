\chapter{Introdução}
	\begin{myenv}{1.5}
		\section{Considerações Iniciais}
			\par Os \textit{voice spoofing attacks} do tipo \textit{playback speech} constituem o tema de estudo deste trabalho.
						 
		\section{Objetivos}
			\par Encontrar um conjunto de características que demonstrem ser as mais disjuntas possíveis para fins de separação entre as classes, com o objetivo de melhorar a acurácia de classificadores para detecção de ataques de \textit{voice spoofing}. Caraterísticas essas com base na transformada \textit{wavelet}, devido a sua boa resolução em relação às dimensões de tempo e frequência. Essas características serão avaliadas usando a análise paraconsistente de acordo com o trabalho \cite{8588433} recentemente publicado.
			
		\section{Estrutura do trabalho}
			\par No capítulo \ref{chap:revBibli} se realizará uma revisão dos seguintes conceitos:
			\begin{itemize}
				\item Engenharia paraconsistente.
				\item Filtros digitais usando wavelets.
				\item Sub-amostragem.
				\item Amostragem, quantização e o formato do arquivo Wave.
				\item Caracterização dos processos de produção da voz humana.
				\item Escalas e energias dos sinais.
			\end{itemize}
			\par Em seguida no capítulo \ref{chap:propApproach} apresentar-se-á a proposta de aproximação do problema usada nesta investigação. 
			\par No capítulo \ref{chap:testsResults} são mostrados os testes e resultados obtidos para, no capítulo \ref{chap:conclusions} se apresentar as conclusões.
			\par Ao final do documento se encontram as referências usadas.
		
		
	\end{myenv}