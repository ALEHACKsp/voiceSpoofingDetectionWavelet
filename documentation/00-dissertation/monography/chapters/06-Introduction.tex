\chapter{Introdução}
	\section{Considerações Iniciais e Objetivos}
		\par A verificação automática de voz, ou biometria de voz, tem atraído cada vez mais a atenção de organizações nacionais e internacionais pois constitui uma alternativa aos sistemas atuais de verificação de usuários, ou ainda, como uma camada adicional de segurança para os sistemas tradicionais. 
		
		\par Em tempos de emergência sanitária nos quais a implementação de sistemas de reconhecimento por imagem de íris tem custos proibitivamente altos e que o reconhecimento por impressões digitais incorre em riscos para a saúde por necessitar de contato com os sensores, o reconhecimento de voz aparece como uma alternativa viável de autenticação.		
		
		\par Assim, com o crescente interesse e uso, consequentemente, crescem também as tentativas de fraude contra tais sistemas. Portanto, é importante que esses consigam diferenciar uma tentativa de autenticação legítima de uma fraudulenta. Considerando tal contexto, os \textit{voice spoofing attacks} do tipo \textit{playback speech} constituem o tema de estudo deste trabalho. 

		\par Particularmente, o objetivo deste trabalho é o de encontrar um conjunto de características que demonstrem ser as mais disjuntas possíveis para fins de separação entre duas classes, isto é, locuções genuínas e falsificadas, com o objetivo de utilizá-las em associação com classificadores e, assim, obter uma técnica eficiente para detectar tentativas de burlar os sistemas de verificação de locutores por voz. As caraterísticas examinadas, que foram obtidas com base na Transformada \textit{Wavelet} devido as boas resoluções em relação às dimensões de tempo e frequência, foram avaliadas usando a Análise Paraconsistente de acordo com o trabalho \cite{8588433}, recentemente publicado. Uma vez avaliadas, os melhores descritores foram isolados para uso em conjunto com duas estruturas de classificação: Uma por distância (Euclidiana ou Manhattan) e outra constituída por uma Máquina de Vetores de Suporte (SVM). 
		
		\par Os resultados dos experimentos descritos neste documento foram apresentados e discutidos detalhadamente, conduzindo a um conjunto de interessantes conclusões, tanto do ponto de vista de processamento digital de sinais quanto de sistemas inteligentes. Acredita-se, assim, que este trabalho forneça uma interessante contribuição, possibilitando ainda futuras investigações. 
	
	\section{Estrutura do trabalho}
		\par No Capítulo \ref{chap:revBibli} é realizada uma revisão dos seguintes conceitos: Amostragem, quantização, formato de arquivos do tipo wave, caracterização dos processos de produção da voz humana, escalas e energias dos sinais, filtros digitais usando \textit{wavelets} e \textit{wavelet-packets}, engenharia paraconsistente de características, além dos trabalhos correlatos contemplando o estado-da-arte. Em seguida, no Capítulo \ref{chap:propApproach}, apresenta-se a abordagem proposta para solucionar o problema considerado nesta investigação. No capítulo \ref{chap:testsResults}, são descritos os testes e os resultados obtidos para que, no Capítulo \ref{chap:conclusions}, apresentem-se as conclusões. Ao final do documento, encontram-se as referências usadas e um apêndice contendo informações complementares em relação ao estudo.