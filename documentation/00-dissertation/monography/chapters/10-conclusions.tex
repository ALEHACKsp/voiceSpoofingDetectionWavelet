\chapter{Conclusões e Trabalhos Futuros} \label{chap:conclusions}
	\par As hipóteses iniciais deste trabalho consideravam que, a decomposição dos sinais de voz no máximo nível possível, usando \textit{wavelets} de qualquer natureza, não alteraria o resultado quando do cálculo de energia do sinal segundo as escalas \textit{BARK} ou \textit{MEL} já que, no final, os intervalos das frequências seriam, de qualquer forma, tratados segundo as regras de cada escala. Após tal fato mostrar-se falso, como constatado nos resultados, que demonstraram ser a combinação \textbf{\textit{wavelet-packet Haar + escala Bark}} a melhor posicionada no plano paraconsistente e \textit{wavelet-packet daubechies 54 + escala Mel} a pior, uma outra hipótese se formou.
	
	\par A segunda hipótese formulada era que as combinações \textit{wavelet + Mel} seriam superiores a \textit{wavelet + Bark}, pois a primeira é otimizada segundo a percepção humana das frequências sonoras. Novamente, esse pensamento se provou incorreto e inconsistente. Analisando os resultados, é possível notar que a escala \textbf{\textit{Bark} é superior a \textit{MEL} por promover um espalhamento horizontal maior dos valores para os vetores de características}, tornando as diferenças das energias calculadas dentro da mesma classe, genuína ou regravada, menores, possibilitando que os vetores de características intraclasses variem menos.

    \par De modo geral, os experimentos mostraram que, \textbf{com ajuda da engenharia paraconsistente de características é possível utilizar classificadores relativamente modestos já que essa metodologia fornece os vetores de características que representam mais propriamente as classes}.
    
    %\par Há que se destacar o grande aprendizado obtido para que fosse possível a realização desta dissertação. A maioria das disciplinas cursadas durante o Mestrado possibilitaram uma nova, e principalmente mais profunda, visão das áreas de interesse do autor, que passou a considerar diferentes formas de ver, pensar e resolver problemas. O curso superou as expectativas tanto em qualidade e conteúdo como em exigência. Desde o cumprimento dos créditos até a finalização, foram inúmeras as horas de estudo e leitura necessárias para entender os assuntos tratados nas disciplinas e transpor algumas deficiências principalmente na base matemática. Milhares de linhas de texto e códigos lidas e escritas, centenas de páginas em artigos e livros desafiaram a perseverança, as energias mentais e algumas vezes até mesmo a resistência física dadas as muitas horas em uma cadeira e as outras tantas de insônia.
    
    \par Em se tratando de uma continuação deste trabalho, a intenção de se explorar os resultados das transformadas \textit{wavelets} até um certo nível somente, em oposição a transformações sucessivas como as que foram utilizadas neste trabalho, ao menos hipoteticamente, parece promissora. Aplicar \textit{wavelets} de forma que fique clara a separação dos componentes de alta e baixa frequência do sinal aparenta ser uma boa base para verificação de \textit{voice spoofing} pois, quando os gráficos representativos para cada nível de transformação são observados, em um certo momento ficam visualmente nítidas as diferenças entre o sinal genuíno e o falseado. No tocante a utilização de escalas e considerando que o objetivo da proposta seja apenas verificar a falsificação ou não de um sinal vocal, novamente hipoteticamente, talvez fosse mais interessante focar a análise em bandas de frequência superiores dado que, durante os estudos dos sinais falseados, os mesmos apareciam 
   ``contaminados'' com componentes de alta frequência.