\chapter{Conclusões}
\label{chap:conclusions}
	\par As hipóteses iniciais neste trabalho consideravam que, a decomposição dos sinais ao máximo por \textit{wavelets} de qualquer natureza não alteraria o resultado quando do cálculo de energia do sinal segundo as escalas \textit{BARK} ou \textit{MEL} já que, no final, os intervalos das frequências seriam, de qualquer forma, tratados segundo as regras de cada escala.
	\par Após o exposto acima se mostrar falso, como constatado nos resultados (\ref{chap:testsResults:sec:Experimento01}) do experimento 1  (\ref{chap:propApproach:sec:Experimento01}) que demonstrou ser a combinação \textbf{\textit{wavelet packet haar + escala BARK} a melhor} posicionada no plano paraconsistente e \textit{wavelet packet daubeechies 54 + escala MEL} a pior, uma outra hipótese se formou após os resultado do experimento 2 e 3.
	\par A segunda hipótese formulada era que as combinações \textit{wavelet + BARK} eram superiores a \textit{wavelet + MEL}, pois a primeira cobria um intervalo de frequências um pouco maior. A ideia era que os ruídos adicionados no sinal por um gravador seriam de frequências, em boa parte, não presentes nas comunicações humanas e que, pela escala \textit{MEL} se basear na audição humana, esta seria inferior na detecção de interferências.
	\par Novamente, esse pensamento se provou incorreto. Analisando os gráficos (\ref{chap:testsResults:sec:Experimento05}) resultantes do experimento 5 (\ref{chap:propApproach:sec:Experimento05}), é possível notar que a escala \textbf{\textit{BARK} é superior a \textit{MEL} por considerar intervalos de frequências menores e em maior quantidade}, o que faz com que as diferenças das energias calculadas dentro da mesma classe (\textit{spoofing} ou \textit{não spoofing}) seja menor fazendo com que os vetores de características variem menos.
	\par No experimento 4 (\ref{chap:propApproach:sec:Experimento04}), não haviam hipóteses iniciais. Esta parte mostrou que a \textbf{transformada \textit{wavelet packet haar}, produz sinais com muito menos flutuações} que as outras wavelets usadas proporcionando uma melhor base para o cálculo das escalas \textit{BARK} e \textit{MEL}.
	\par Os experimentos 2 (\ref{chap:propApproach:sec:Experimento02}) e 3 (\ref{chap:propApproach:sec:Experimento03}) mostram nos seus respectivos resultados (\ref{chap:testsResults:sec:Experimento02} e \ref{chap:testsResults:sec:Experimento03}) que, \textbf{com ajuda da engenharia paraconsistente de características é possível fazer com que classificadores relativamente simples tenham desempenhos satisfatórios}.