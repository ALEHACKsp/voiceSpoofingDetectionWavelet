\begin{frame}
	\frametitle{Procedimento 01}
		\only<1>{
			\framesubtitle{Características}
				\begin{itemize}
					\item MEL: 13 coeficientes.
					\item BARK: 24 coeficientes.
					\item Redimensionar o sinal.
					\item Determinar o máximo de transformações.
					\item Transformadas wavelet-packet até o nível máximo.
					\item Selecionar a melhor combinação \textit{wavelet}/escala.
				\end{itemize}
		}
		\only<2>{
			\framesubtitle{Comprimento e máximo de transformações do sinal}
			\par Determinação do tamanho ótimo para as transformações
			\begin{equation}
				tamanhoOtimo=2^{proxInt(\log_{2}tamanho)}
				\label{eq:optimalSize}
			\end{equation} 
			
			\par Quantidade máxima de transformações
			\begin{equation}
				maxTrans=\log_{2}(tamanho) \qquad.
				\label{eq:maxWaveletTransf}
			\end{equation}
		}
		\only<3>{
			\framesubtitle{Algoritmo}
			\begin{lstlisting}[language=C++]
// Carregue para a memoria um dos conjuntos de amostra
for (listaDeAmostras : {listaComVoiceSpoofing, listaSemVoiceSpoofing}) {
	// Selecione o proximo tipo de wavelet
	for (wavelet : wavelets) {
		// Selecione entre BARK ou MEL
		for (barkOuMel : {BARK, MEL}) {
			// Selecione o proximo sinal dentro da amostra
			for (sinal : listaDeAmostras) {
				tamanhoOtimo=calcularTamanhoOtimo(sinal);
				redimensionar(sinal, tamanhoOtimo);
				sinalTransformado=wavelet(sinal, wavelet);
				energias=calcularEnergias(sinalTransformado, barkOuMel);
				energias=normalizar(energias);
				
				// Armazene os resultados
				resultados[wavelet.nome()][barkOuMel][listaDeAmostras.nome()].adicionar(energias);
			}
		}
	}
}
// Posicione os resultados no plano paraconsistente
mostraResultadosNoPlanoParaconsistente(resultados);
\end{lstlisting}
		}
\end{frame}

\begin{frame}
	\frametitle{Procedimento 02}
	\only<1>{
		\framesubtitle{Características}
		\begin{itemize}
			\item Bandas críticas: BARK.
			\item Wavelet: Haar.
			\item Modelos de 10\% 20\% 30\% 40\% e 50\%.
			\item Sorteio aleatório de vetores de características.
			\item Verifica a acurácia e o EER de classificadores \textit{pattern-matching} por distâncias Euclidiana e Manhattan.
		\end{itemize}
	}
	\only<2>{
		\framesubtitle{Algoritmo}
		\begin{lstlisting}[language=C++, caption={Procedure 02 algorithm}, label={lst:experiment02Algo}]
modelProportion={0.1, 0,2, 0,4, 0,5};
genuineModel = spoofingModel= genuineTest = spoofingTest = accuracyList = {};
for (distance : {Euclidian, Manhattan}) {
	for(percentage : modelProportion){
		for(testCounter = 0; testCounter < 300; testCounter++){
			// Choose feature vectors randomly from spoofing set
			// to build the model according to the proportions chosen.
			chooseRamdomly(voiceSpoofingSet, percentage, spoofingModel, spoofingTest);
			// Choose feature vectors randomly from genuine set
			// to build the tests according to the proportion chosen
			chooseRamdomly(genuineSet, percentage, genuineModel, genuineTest);
			
			trainClassificator("spoofing", spoofingModel);
			trainClassificator("genuine", genuineModel);
			// Classify the spoofing tests against the 
			// spoofing model and fill the confusion tables
			for(signal : spoofingTest){
				fillConfusionTable(signal, "spoofing");
			} 
			// Classify the genuine tests against the
			// genuine model and fill the confusion tables
			for(signal : genuineTest){
				fillConfusionTable(signal, "genuine");
			}

			accuracy=calculateAccuracy();
			
			// Store the accuracies for each percentage
			accuracyList[percentage].add(accuracy);

			// Store the best accuracy and its respective confusion table
			if(isTheBestAccuracy(accuracy)){
				saveAccuracyAndItsConfusionTable();
			}
			
			// Store the worst accuracy and its respective confusion table
			if(isTheWorstAccuracy(accuracy)){
				saveAccuracyAndItsConfusionTable();
			}
		}
		
		// Calculate and save the standard deviation for current proportion
		calculateAndSaveTheStandardDeviation(accuracyList[percentage]);
	}
}				
\end{lstlisting}
	}
\end{frame}

\begin{frame}
	\frametitle{Procedimento 03}
	\only<1>{
		\framesubtitle{Características}
		\begin{itemize}
			\item Bandas críticas: BARK.
			\item Wavelet: Haar.
			\item Modelos de 10\% 20\% 30\% 40\% e 50\%.
			\item Sorteio aleatório de vetores de características.
			\item Verifica a acurácia e o EER de uma \textit{Support Vector Machine (SVM)}.
		\end{itemize}
	}
	\only<2>{
		\framesubtitle{Características da SVM}
		\begin{figure}
	\centering
	\scalebox{2}{
				\begin{tikzpicture}
	%input layer
	\node (input1) at (0.65,2.0) {};
	\draw[->,in=180,out=0] (1.0,2.0) to (1.35,2.0);
	\node (input2) at (0.65,1.5) {};
	\draw[->,in=180,out=0] (1.0,1.5) to (1.35,1.5);
	\node (input3) at (0.65,1.0) {};
	\draw[->,in=180,out=0] (1.0,1.0) to (1.35,1.0);
	\node (input3) at (0.65,0.0) {};
	\draw[->,in=180,out=0] (1.0,0.0) to (1.35,0.0);
	\node (IL_N1) at (1.5,2.0) {}; \filldraw[fill=gray!30] (1.5,2.0) circle (0.15cm);
	\node (IL_N2) at (1.5,1.5) {}; \filldraw[fill=gray!30] (1.5,1.5) circle (0.15cm);
	\node (IL_N3) at (1.5,1.0) {}; \filldraw[fill=gray!30] (1.5,1.0) circle (0.15cm);
	\node at (1.5,0.625) {$\vdots$};
	\node (IL_Nn) at (1.5,0.0) {}; \filldraw[fill=gray!30] (1.5,0.0) circle (0.15cm);
	
	%hidden layer 
	\node (HL_N1) at (3.5,2.5) {}; \filldraw[fill=blue!20] (3.55,2.5) circle (0.15cm);
	\node (HL_N2) at (3.5,2.0) {}; \filldraw[fill=blue!20] (3.55,2.0) circle (0.15cm);
	\node (HL_N3) at (3.5,1.5) {}; \filldraw[fill=blue!20] (3.55,1.5) circle (0.15cm);
	\node (HL_N4) at (3.5,1.0) {}; \filldraw[fill=blue!20] (3.55,1.0) circle (0.15cm);
	\node at (3.55,0.625) {$\vdots$};
	\node (HL_Nn-1) at (3.5,0.0) {}; \filldraw[fill=blue!20] (3.55,0.0) circle (0.15cm);	
	\node (HL_Nn) at (3.5,-0.5) {}; \filldraw[fill=blue!20] (3.55,-0.5) circle (0.15cm);	
	\draw[->,in=180,out=0] (IL_N1)+(1.5mm,0) to (HL_N1);
	\draw[->,in=180,out=0] (IL_N1)+(1.5mm,0) to (HL_N2);
	\draw[->,in=180,out=0] (IL_N1)+(1.5mm,0) to (HL_N3);
	\draw[->,in=180,out=0] (IL_N1)+(1.5mm,0) to (HL_N4);
	\draw[->,in=180,out=0] (IL_N1)+(1.5mm,0) to (HL_Nn-1);
	\draw[->,in=180,out=0] (IL_N1)+(1.5mm,0) to (HL_Nn);
	
	\draw[->,in=180,out=0] (IL_N2)+(1.5mm,0) to (HL_N1);
	\draw[->,in=180,out=0] (IL_N2)+(1.5mm,0) to (HL_N2);
	\draw[->,in=180,out=0] (IL_N2)+(1.5mm,0) to (HL_N3);
	\draw[->,in=180,out=0] (IL_N2)+(1.5mm,0) to (HL_N4);
	\draw[->,in=180,out=0] (IL_N2)+(1.5mm,0) to (HL_Nn-1);
	\draw[->,in=180,out=0] (IL_N2)+(1.5mm,0) to (HL_Nn);
	
	\draw[->,in=180,out=0] (IL_N3)+(1.5mm,0) to (HL_N1);
	\draw[->,in=180,out=0] (IL_N3)+(1.5mm,0) to (HL_N2);
	\draw[->,in=180,out=0] (IL_N3)+(1.5mm,0) to (HL_N3);
	\draw[->,in=180,out=0] (IL_N3)+(1.5mm,0) to (HL_N4);
	\draw[->,in=180,out=0] (IL_N3)+(1.5mm,0) to (HL_Nn-1);
	\draw[->,in=180,out=0] (IL_N3)+(1.5mm,0) to (HL_Nn);
	
	\draw[->,in=180,out=0] (IL_Nn)+(1.5mm,0) to (HL_N1);
	\draw[->,in=180,out=0] (IL_Nn)+(1.5mm,0) to (HL_N2);
	\draw[->,in=180,out=0] (IL_Nn)+(1.5mm,0) to (HL_N3);
	\draw[->,in=180,out=0] (IL_Nn)+(1.5mm,0) to (HL_N4);
	\draw[->,in=180,out=0] (IL_Nn)+(1.5mm,0) to (HL_Nn-1);
	\draw[->,in=180,out=0] (IL_Nn)+(1.5mm,0) to (HL_Nn);
	
	%output layer
	\node (OL_N1) at (5.5,1.0) {}; \filldraw[fill=red!40] (5.55,1.0) circle (0.15cm);
	
	\draw[->,in=180,out=0] (HL_N1)+(2mm,0) to (OL_N1);
	\draw[->,in=180,out=0] (HL_N2)+(2mm,0) to (OL_N1);
	\draw[->,in=180,out=0] (HL_N3)+(2mm,0) to (OL_N1);
	\draw[->,in=180,out=0] (HL_N4)+(2mm,0) to (OL_N1);
	\draw[->,in=180,out=0] (HL_Nn-1)+(2mm,0) to (OL_N1);
	\draw[->,in=180,out=0] (HL_Nn)+(2mm,0) to (OL_N1);
	%
	\draw[snake=brace,mirror snake,raise snake=45pt,brown] (1.25,1.25) -- (1.75,1.25) node[black,midway,yshift=-50pt,below]{\tiny camada de} node[black,midway,yshift=-58pt,below]{\tiny entrada com} node[black,midway,yshift=-66pt,below]{\tiny $R$ elementos}
	node[black,midway,yshift=-74pt,below]{\tiny passivos};
	\draw[snake=brace,mirror snake,raise snake=45pt,brown] (3.25,0.75) -- (3.75,0.75) node[black,midway,yshift=-50pt,below]{\tiny camada} node[black,midway,yshift=-58pt,below]{\tiny intermediária} node[black,midway,yshift=-66pt,below]{\tiny com $X$ neurônios}
	node[black,midway,yshift=-74pt,below]{\tiny ativos não-lineares};
	\draw[snake=brace,mirror snake,raise snake=45pt,brown] (5.25,1.75) -- (5.75,1.75) node[black,midway,yshift=-50pt,below]{\tiny camada de} node[black,midway,yshift=-58pt,below]{\tiny saída com} node[black,midway,yshift=-66pt,below]{\tiny um elemento}
	node[black,midway,yshift=-74pt,below]{\tiny ativo linear};
	%
	\node (OUT) at (6.5,1.0) {\tiny resultado}; 
	\draw[->,in=180,out=0] (OL_N1)+(2mm,0) to (OUT);	
	
	\node at (4,2.6) {\tiny $p_0$}; 
	\node at (4,2.1) {\tiny $p_1$}; 
	\node at (4,1.65) {\tiny $p_2$}; 
	\node at (4,1.2) {\tiny $p_3$}; 
	\node at (4,0.2) {\tiny $p_{X-2}$}; 
	\node at (4,-0.3) {\tiny $p_{X-1}$}; 
\end{tikzpicture}
	}
	\caption{Estrutura da SVM para o procedimento 03 com $R$ neurônios na camada de entrada, sendo $R$ a dimensão dos vetores de características, e $X$ neurônios na camada intermediária, sendo $X$ o número de casos de treinamento}
	\label{fig:3layersSVM}
\end{figure} 
	}
	\only<3>{
		\framesubtitle{Características da SVM}
		\begin{itemize}
			\item Três camadas: Entrada, segunda com elementos ativos não-lineares de núcleos Gaussianos e saída; 
			\item Inexistem pesos entre a camada de entrada e a camada intermediária;
			\item A saída de cada elemento da camada intermediária conecta-se com o único elemento da camada de saída por meio dos pesos $p_0, p_1, .... p_{X-1}$;
			\item O valor de saída consiste na combinação linear dos pesos com os valores recebidos como entrada da camada de saída;
			\item Solução direta de um sistema linear quadrado, isto é, possível e determinado.
		\end{itemize}
		
		\par Todos os arranjos para a seleção dos vetores de treinamento e testes, assim como demais detalhes, são idênticos àqueles do procedimento 02.
	}
	\only<4>{
		\framesubtitle{Algoritmo}
		\begin{lstlisting}[language=C++, caption={Algoritmo que caracteriza o procedimento 03}, label={lst:experiment03Algo}]
tamanhosDoModelo={0.1, 0,2, 0,4, 0,5};
modeloDeReferenciaNaoSpoofing={};
modeloDeReferenciaSpoofing={};
testesNaoSpoofing={};
testesSpoofing={};
	for(porcentagem : tamanhosDoModelo){
		for(teste = 0; teste < 300; teste++){
			// Escolhe aleatoriamente os sinais para o modelo com spoofing 
			// e os grava em 'modeloDeReferenciaSpoofing' o restante vai 
			// para 'testesSpoofing'
			escolherAleatoriamente(listaComVoiceSpoofing, porcentagem, modeloDeReferenciaSpoofing, testesSpoofing);
			
			// Escolhe aleatoriamente os sinais para o modelo sem spoofing
			// e os grava em 'modeloDeReferenciaNaoSpoofing' o restante vai 
			// para 'testesNaoSpoofing'
			escolherAleatoriamente(listaSemVoiceSpoofing, porcentagem, modeloDeReferenciaNaoSpoofing, testesNaoSpoofing);
			treinarClassificador("spoofing", modeloDeReferenciaSpoofing);
			treinarClassificador("naoSpoofing", modeloDeReferenciaNaoSpoofing);
			
			// Classifica os testes e preenche a tabela de confusao
			for(sinal : testesSpoofing){
				preencherTabelaDeConfusao(sinal, "spoofing");
			} 
			
			// Classifica os testes e preenche a tabela de confusao
			for(sinal : testesNaoSpoofing){
				preencherTabelaDeConfusao(sinal, "naoSpoofing");
			}
			
			acuracia=calculaAcuracia();
			
			// Salva a melhor acuracia e matriz de confusao
			if(ehAMelhorAcuracia(acuracia)){
				salvaAcuraciaEMatrizDeConfusao();
			}
			
			// Salva a pior acuracia e matriz de confusao
			if(ehAPiorAcuracia(acuracia)){
				salvaAcuraciaEMatrizDeConfusao();
			}
		}
	}			
\end{lstlisting}
	}
\end{frame}