\section{Conclusions}
\label{sec:conclusions}
\par In this paper, a wavelet-based strategy to distinguish between genuine and spoofed speech was described. Together with inputs drawn  from the auditory Bark and Mel scales, feature vectors assessed with PFE were successfully used in conjunction with either two metric distances or an SVM, allowing for a value of accuracy higher than $95\%$ and an EER equals to 0.102439. Particularly, Haar wavelets associated with Bark scale presented promising results, certifying the efficacy of the proposed approach. Interestingly, although Mel scale was particularly designed to work with speech and voice signals, it showed a degraded result in comparison to Bark scale, according to PFE. Overall, our experiments show that, by using PFE for feature ranking, modest classifiers presented relevant results, confirming that, once features have been consistently selected, strong classifiers can be, sometimes, replaced by modest ones. This is the case of the proposed approach. For more information and access to the source code and database used in this work please read the README.md file at \href{https://github.com/ensismoebius/voiceSpoofingDetectionWavelet}{\textbf{https://github.com/ensismoebius/voiceSpoofingDetectionWavelet}}. All the files used in our experiments, including the database of speech signals, can be downloaded from that Internet address. 
