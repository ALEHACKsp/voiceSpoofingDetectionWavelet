\section{Conclusions and Future Work}
\label{chap:conclusions}

\par The initial hypotheses of this work considered that the decomposition of the voice signals at the maximum possible level, using \textit{wavelets} of any nature, would not change the result when calculating the signal energy according to the \textit{BARK} or scales \textit{MEL} since, in the end, the frequency ranges would, in any case, be treated according to the rules of each scale. After this fact proved to be false, as verified in the results, which demonstrated that the combination \textbf{\textit{wavelet-packet Haar Bark scale}} was the best positioned in the paraconsistent plane and \textit{wavelet-packet daubechies 54 scale Mel} the worst, another hypothesis was formed.

\par The second hypothesis formulated was that the \textit{wavelet Mel} combinations would be superior to \textit{wavelet Bark}, since the first is optimized according to the human perception of sound frequencies. Again, this thought proved to be incorrect and inconsistent. Analyzing the results, it is possible to notice that the scale \textbf{\textit{Bark} is superior to \textit{MEL} for promoting a greater horizontal spreading of the values for the characteristic vectors}, making the differences in the energies calculated within the same class, genuine or rewritten, smaller, allowing the vectors of intraclass characteristics to vary less.

\par In general, the experiments showed that, \textbf{with the help of paraconsistent feature engineering, it is possible to use relatively modest classifiers when the feature vectors are more properly representing the classes}.

\par In the case of a continuation of this work, the intention of exploring the results of the \textit{wavelets} transformations to a certain level only, as opposed to successive transformations such as those used in this work, at least hypothetically, seems promising. Applying \textit{wavelets} in such a way that the separation of the high and low frequency components of the signal is clear seems to be a good basis for checking \textit{voice spoofing} because, when the representative graphics for each transformation level are observed, at a certain moment, the differences between the genuine and the re-recorded signal are visually clear. Regarding the use of scales and considering that the objective of the proposal is only to verify the falsification or not of a vocal signal, again hypothetically, it might be more interesting to focus the analysis on higher frequency bands, since, during the studies of the re-recorded signals, they appeared ``contaminated'' with high frequency components.
