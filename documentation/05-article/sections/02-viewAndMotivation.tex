\section{Brief Review and Motivation}
\label{review}
		\par In particular, the objective here is to find a set of characteristics that prove to be the most disjunctive possible for separating two classes, that is, genuine and spoofed signals, to use them in association with classifiers and, thus, obtain an efficient technique to detect attempts to circumvent the verification systems of voice announcers. The characteristics examined, which were obtained based on the Wavelet Transform due to the good resolutions regarding the dimensions of time and frequency, were evaluated using Paraconsistent Analysis according to the work \cite{8588433}, recently published. Once evaluated, the best descriptors were isolated for use in conjunction with two classification structures: One by distance (Euclidean or Manhattan) and the other consisting of a Support Vector Machine (SVM).

		\par The experiments results described in this document were presented and discussed in detail, leading to a set of interesting conclusions, both from the point of view of digital signal processing and intelligent systems. It is believed, therefore, that this work provides an interesting contribution, making possible future investigations. 
		
		\subsection{Work structure}
			%\par The section \ref{chap:revBibli} reviews the following concepts: Sampling, quantization, wave file format, characterization of human voice production processes, signal scales and energies, digital filters using wavelets and wavelet-packets, paraconsistent engineering of characteristics, in addition to related works contemplating the state-of-the-art.
			\par In section \ref{sec:propApproach}, the proposed approach to solve the problem considered in this investigation is presented. In the section \ref{sec:testsResults}, the tests and the results obtained are described so that, in section \ref{sec:conclusions}, the conclusions are presented. At the end of the document, there are the references used and an appendix containing additional information about the study.